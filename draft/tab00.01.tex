\documentclass{standalone}
\usepackage{ctex}
\usepackage{tikz}
\usetikzlibrary{positioning,calc,decorations.pathreplacing}

\setmainfont{XITS}
\setCJKmainfont{FZShuSong-Z01}
  [
    BoldFont={Source Han Serif SC Bold},
    ItalicFont=FZKai-Z03
  ]

\newcommand{\pbox}[2][4em]{\parbox{#1}{\baselineskip=0pt\centering #2}}

\begin{document}
\begin{tikzpicture}[x=1em,y=1em,anchor=west,line width=0.8pt,inner sep=0.5em,outer sep=0]
\node (sky)    at (0,22) {天空};
\node (ground) at (0,20) {地面};
\node (time)   at (0,18) {时间};
\node (space)  at (0,16) {空间};
\node (elec)   [text width=2em] at (0,14) {电};
\node (magn)   [text width=2em] at (0,12) {磁};
\node (light)  [text width=2em] at (0,10) {光};
\node (atom)   at (0,8) {原子};
\node (nucl)   at (0,6) {核子};
\node (quark)  at (0,3) {夸克};
\node (prim)   at (0,0) {物原?};
\coordinate (prim2) at ($(prim.west)+(3,0)$);

\draw [decorate,decoration=brace]
  (sky.east) -- (ground.east);
\node (newt)
  at ($(sky.east)!0.5!(ground.east)$)
  {\pbox{牛顿力学(1687)}};
\draw [decorate,decoration=brace]
  (elec.east) -- (light.east);
\node (elecdyna)
  at ($(elec.east)!0.5!(light.east)$)
  {\pbox{电动力学(1867)}};

\draw [decorate,decoration=brace]
  (newt.east) -- (elecdyna.east);
\node (specrela)
  at ($(newt.east)!0.5!(elecdyna.east)$)
  {\pbox{狭义相对论\,(1905)}};
\node (quanmech)
  at ($(atom.east)!0.5!(nucl.east)+(5,0)$)
  {\pbox{量子力学(1925)}};

\coordinate (newt2) at ($(newt.east)+(5,0)$);
\draw [decorate,decoration=brace]
  (newt2.east) -- (specrela.east);
\node (generela)
  at ($(newt2.east)!0.5!(specrela.east)$)
  {\pbox{广义相对论\,(1916)}};
\draw [decorate,decoration=brace]
  (specrela.east) -- (quanmech.east);
\node (quanelec)
  at ($(specrela.east)!0.5!(quanmech.east)$)
  {\pbox{量子电动力学(1948)}};
\node (weak)
  at ($(nucl.east)!0.5!(quark.east)+(10,0)$)
  {\pbox{弱相互作用\,(1958)}};
\node (strong) [text width=4em]
  at ($(quark.east)!0.5!(prim2.east)+(10,0)$)
  {\pbox[5em]{强相互作用(1934\textasciitilde ?)}};

\draw [decorate,decoration=brace]
  (quanelec.east) -- (weak.east);
\node (weakelec)
  at ($(quanelec.east)!0.5!(weak.east)$)
  {\pbox{弱电统一理论(1967)}};

\coordinate (strong2) at ($(strong.east)+(5,0)$);
\draw [decorate,decoration=brace]
  (weakelec.east) -- (strong2.east);
\coordinate (generela2) at ($(generela.east)+(10.5,0)$);
\node (great)
  at ($(weakelec.east)!0.5!(strong2.east)$)
  {\pbox{大统一理论\,(1970\textasciitilde ?)}};

\coordinate (great2) at ($(great.east)+(0.5,0)$);
\draw [decorate,decoration=brace]
  (generela2.east) -- (great2.east);
\node (super)
  at ($(generela2.east)!0.5!(great2.east)$)
  {\pbox[5em]{超统一理论超引力~~(?)}};
\end{tikzpicture}
\end{document}