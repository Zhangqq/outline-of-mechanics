\documentclass[../outline-of-mechanics.tex]{subfiles}

\begin{document}

\section{加速度的相对性}\label{sec:02.04}

根据速度合成公式\eqref{eqn:02.03.06},如果在时刻$t$,质点对于$K$,$K'$
的速度分别为$\vec{v}\left(t\right)$及$\vec{v}'\left(t\right)$,而$K'$相对于$K$以$\vec{u}$作匀速运动,
则有
\begin{equation}\label{eqn:02.04.01}
  \vec{v}'\left(t\right)=\vec{v}\left(t\right)-\vec{u}
\end{equation}

\clearpage\noindent
根据加速度的定义。质点相对K的加速度是
\begin{equation}\label{eqn:02.04.02}
  \vec{a}\equiv\frac{\dif \vec{v}}{\dif t}
\end{equation}
而相对于$K'$的加速度是
\begin{equation}\label{eqn:02.04.03}
  \vec{a}'\equiv\frac{\dif \vec{v}'}{\dif t}
\end{equation}
将式\eqref{eqn:02.04.01}对时间求导,得
\begin{equation}
  \frac{\dif \vec{v}'}{\dif t}=\frac{\dif \vec{v}}{\dif t}-\frac{\dif \vec{u}}{\dif t}
\end{equation}
因$ \vec{u} $是不随时间变化的,故$\dfrac{\dif \vec{u}}{\dif t}=0$,所以得
\begin{equation}\label{eqn:02.04.04}
  \vec{a}'=\vec{a}
\end{equation}
这个结果告诉我们,同一质点对于两个相互匀速运动的参考系的
加速度是一样的。换言之,质点的加速度对于相对匀速运动的所
有参考系,具有不变性。

最后指出,对于相对以非匀速运动的两个参考系,式\eqref{eqn:02.04.01}
不再成立。例如,若$K'$相对于$K$作匀加速运动。加速度为$\vec{a}_0$,则
$K'$相对于$K$的速度为$\vec{u}=\vec{a}_0t$,从而式\eqref{eqn:02.04.01}相应改为
\begin{equation*}
  \vec{v}'\left(t\right)=\vec{v}\left(t\right)-\vec{a}_0t
\end{equation*}
由此可以推得
\begin{equation}\label{eqn:02.04.05}
  \vec{a}'=\vec{a} - \vec{a}_0
\end{equation}
这表示对于相对以匀加速度$\vec{a}_0$运动的两个参考系,质点对$K$及$K'$
的加速度$\vec{a}$及$\vec{a'}$,二者之间满足矢量加法关系(式\eqref{eqn:02.04.05}).在这
种情况,加速度是没有不变性或绝对性的。

\end{document}
