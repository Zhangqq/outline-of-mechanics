\documentclass[../outline-of-mechanics.tex]{subfiles}

\begin{document}

\section[长度]{\makebox[5em][s]{长度}}\label{sec:01.03}

长度是空间的一个基本性质。

对长度的测量,在日常的范围中,是用各种各样的尺,如米
尺、千分尺、螺旋测微计等等。对于不能用尺直接加以测量的小
尺度,可以求助于光学方法。在精密机床上常有光学测量装置;
测定胰岛素中原子的位置,是用X光衍射方法。对于大的尺度,
也不能直接用尺去测量,也要求助于光。测量月亮与地球的距离
可以用激光测距的方法;测量一些不太远的恒星,可以用三角学
方法,利用恒星发出的光。至于银河系之外的遥远天体的距离,
同样是用它们发光的一些特征来测定的。

最近,长度的单位和标准,也用光来规定了。

长度的位单是米。1960年以前,用铂铱米尺作为标准尺,规
定米的大小。1960年以后,改用光的波长作为标准。在第十一届
国际计量大会上,正式通过的“米”的定义是l米等于$^{86}$Kr原子
\clearpage\noindent
的$2\rm{p}^{10}$和$5\rm{d}^5$ % \errnote{$5\rm{d}^5$}{原文误作“$6\rm{d}_5$”。}
能级之间跃迁时所对应的辐射在真空中的波长$\lambda$的1,650,763.73倍,即
\begin{equation*}
  1 \text{米} = 1,650,763.73 \, \lambda
\end{equation*}

1983年10月召开的第十七届国际计量大会上已正式通过了
\begin{table}[!h]
  \centering
  \caption{一些典型物理现象的空间尺度}
  \label{tab:01.03}
  \begin{tabular*}{\linewidth}{>{\centering}m{\linewidth}c}
    \toprule
    \includegraphics[width=0.8\linewidth]{figure/tab01.03} & \\
    \bottomrule
  \end{tabular*}
\end{table}
\clearpage

\noindent 新的米的定义,即用光速值来定义“米”,以代替1960年的规定。
新的米的定义是,米是光在真空中在$ 1/299,792,458 $秒的时间间隔
内所传播的路程长度。按这种新的定义,光速$ c $是一固定的常数,即
\begin{equation*}
  c = 299,792,458 \, \text{米/秒}
\end{equation*}

表\ref{tab:01.03}中列举了一些典型现象的空间尺度。目前,物理学中涉
及的最大长度是$10^{28}$米,它是宇宙曲率半径的下限;已达到的最
小长度为$10^{-20}$米,它是弱电统一的特征尺度。普朗克长度约为
$10^{-35}$米,被认为是最小的长度,意思是说,在比普朗克长度更小
的范围内,长度的概念可能就不再适用了。

\end{document}
