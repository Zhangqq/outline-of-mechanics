\documentclass[../outline-of-mechanics.tex]{subfiles}

\begin{document}

\section{狭义相对论的基本原理}\label{sec:11.03}

爱因斯坦首先认识到相对性原理的特别的重要性。他强调这
是整个物理学都必须遵守的一条最基本的原则,也就是不仅是力
学规律,而且任何动力学规律,对不同惯性系应具有相同的形
式;不仅用力学现象,而且用任何现象都不能测出绝对速度。

诚然,在伽利略的原始表述中所列举的种种现象(水滴、蝶
飞、鱼游……)中,并非只限于力学现象。但是,在牛顿力学时
期,受到定量研究的现象只有力学现象。因此,相对性原理实际
% 323.jpg
上只是相当于要求牛顿动力学规律适用于所有惯性系。因此,它
并没有带来什么新结果。这一点在上节已讨论过了。

然而,如果把相对性原理也应用于光学现象,立即会得到不
同于牛顿理论的新结果。为此,我们回忆一下第二章的一些结
果。我们曾利用一个特制的雷达钟,并根据光速不变性质,推得
了运动钟的变慢。并且说这种变慢现象与所用的钟无关,是普
遍的,即用其他钟来测量也一样。这个断言就是根据相对性原
理。如果时钟变慢只是雷达钟的结果,那么,我们就可以找到一
个“好”钟,它在K中和K中总是走得一样,没有变慢现象。这
样,当“好”钟与雷达钟都在K中时,我们将看到它们走得一样
快慢,即同步的。而当它们在K中时,就不一样了,因为“好”
钟仍与K中的钟一样快慢,而雷达钟不同,所以两钟必有偏差。
由此,我们就可以从两钟的偏差来表明各惯性系不是平权的、等
同的,而这是与相对性原理相矛盾的,因此,相对性原理要求,
只要有一种钟变慢,则其他钟必然也变慢。

上述例子已使我们看到,从相对性原理及光速不变性,能得
出多么重要的不同于牛顿经典力学的结论。爱因斯坦所发展的狭
义相对论,就是以这两条为出发点,根本改造了牛顿理论的时空
观。本章下面几节,将较系统地讨论狭义相对论的主要内容。在
此之前,我们再用爱因斯坦的原话来表述一下狭义相对论的两条
原理。1905年,爱因斯坦在他的狭义相对论的奠基性论文《论运
动物体的电动力学》一文中写道:

\begin{quoting}
  “下面的考虑是以相对性原理和光速不变原理为依
  据的,这两条原理我们规定如下:

  (1)物理体系的状态据以变化的定律,同描述这些状
  态变化时所参照的坐标系究竟是用两个在互相匀速移动
  着的坐标系中的哪一个并无关系。

  (2)任何光线在静止的坐标系中都是以确定的速
  % 324.jpg
  度$ c $运动着,不管这道光线是由静止的还是运动的物体
  发射出来的。”
\end{quoting}

应当说明,对于任何普遍性的原理,我们在原则上总是不能
说实验证明了这个原理,因为普遍的原理总是涉及无限多的具体
情况,而在有限的时间里,我们只能完成涉及有限具体情况的实
验。因此,与其说用实验去证明某原理,不如说用实验去验证从
该原理所推得的种种具体结论。我们也将以这种态度来对待狭义
相对论的两条基本原理。所以,我们将研究由这两条原理出发,
到底能得到哪些特别的结论。

\end{document}
