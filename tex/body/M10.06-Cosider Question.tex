\documentclass[../outline-of-mechanics.tex]{subfiles}

\begin{document}

\begin{questions}

  \question 计算物体的转动惯量时,可否将物体的质量看作集中在质心处?

  \question 设有两个圆盘用密度不同的金属制成,但重量和厚度都相
  同,哪个圆盘有较大的转动惯量?

  \question 图\ref{fig:10.24} 中是五个物体的横截面,这些截面具有相等的高
  度和相等的最大宽度,这些物体又都具有相等的质量。对于与截
  面垂直且经过几何中心的转轴来说,哪个物体的转动惯置最大?
  哪个最小?

  \vspace{1.5em}
  \begin{figure}[h]
    \centering
    \includegraphics{figure/fig10.24}
    \caption{}
    \label{fig:10.24}
  \end{figure}

  \question 本章\ref{sec:10.02} 节中指出,为了描写刚体的一般运动,可将运动
  分解为随某选定的基点$ O $的平动加绕$ O $的转动,平动的距离及
  \clearpage\noindent
  速
  % 311.jpg
  度随基点的不同选择而有所不同,但转过的角度及角速度是与基
  点的选择无关的,这称之为角速度的绝对性。试问:这基点可以
  \CJKunderdot{任意}选择吗?可以选在刚体之外的空间某点上吗?

  \question 有一均匀的实心圆柱体,沿着一光滑斜面自由落下,问它
  滑下时与液下时其末速度是否相等?

  \question 在光滑水平面上有一均匀细棒。证明:在力偶作用下,不
  管这力偶作用于棒的哪一部位,棒的质心加速度均为零。

\end{questions}

\end{document}
