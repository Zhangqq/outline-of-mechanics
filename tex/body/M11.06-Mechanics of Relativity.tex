\documentclass[../outline-of-mechanics.tex]{subfiles}

\begin{document}

\section{相对论力学}\label{sec:11.06}

现在我们来讨论经典力学规律与新的时空观的关系。牛顿运
动方程与相对论时空观是有矛盾的。例如,相对论时空观断言,
不能把物体加速到超过光速。但是,按照牛顿力学,在原则上总
是可以把任何物体加速到大于光速,只要对该物体施以长时间的
足够大的力。因此,牛顿运动方程必须加以修正。修正的原则是
相对性原理。

我们要寻找能与狭义相对论相协调的力学规律。注意到在低
速范围洛伦兹变换过渡为伽利略变换,所以相对论力学规律,在
低速范围应过渡为牛顿运动方程。根据这一考虑,我们仍可以将
相对论中的力学规律写成

\begin{equation}\label{eqn:11.06.01}
  \begin{split}
    \frac { d } { d t } \left( m v _ { x } \right) &= F _ { x } \\
    \frac { d } { d t } \left( m v _ { y } \right) &= F _ { y } \\
    \frac { d } { d t } \left( m v _ { z } \right) &= F _ { z }
  \end{split}
\end{equation}
但是,不同于牛顿情况,现在质量$ m $及力$ F $不再是绝对量,即相
对于不同的坐标系,其数值是不同的。只当速度小时,这二者才
是绝对量。我们不去全面地证明方程\eqref{eqn:11.06.01}如何符合狭义相对
论的要求。下面仅讨论式\eqref{eqn:11.06.01}若遵从狭义相对论,质量$ m $应
怎样变换,为了区别,以后称牛顿理论中所用的质点质量为静止质
量$ m _ 0 $。

考虑两个质点。相对于$ K $,它们的质量分别为$ m_1 $及$ m_2 $,速度
都在$ x $方向上,分别为$ u_1, u_2 $,并且两质点的动量和为零。
\begin{equation}\label{eqn:11.06.02}
  P = m _ { 1 } u _ { 1 } + m _ { 2 } u _ { 2 } = 0
\end{equation}
两质点所构成的体系的总质量为
\begin{equation}\label{eqn:11.06.03}
  M = m _ { 1 } + m _ { 2 }
\end{equation}
而整个体系对K的速度,按定义为$ V = \dfrac { P } { M } = 0 $,即静止。

再从$ K ' $中的观察者来分析这两个质点。根据相对性原理,
如果力学规律具有式\eqref{eqn:11.06.01}的形式,则两质点对$ K ' $的动量仍
应具有式\eqref{eqn:11.06.02}的形式,即
\begin{equation}\label{eqn:11.06.04}
  P ^ { \prime } = m _ 1 ^ { \prime } u _ 1 ^ { \prime } + m _ 2 ^ { \prime } u _ 2 ^ { \prime }
\end{equation}
所有带撇量均表示对K系而言的。这时体系的总质量为:
\begin{equation}\label{eqn:11.06.05}
  M ^ { \prime } = m _ 1 ^ { \prime } + m _ 2 ^ { \prime }
\end{equation}
再则,由于$ K ' $相对于$ K $以速度$ v $运动,所以对$ K $为静止的体系,
对$ K ' $有速度$ -v $。这样,两质点所构成的整个体系对$ K ' $的速度应
为$ -v $。所以动量$ P ' $又可写成
\begin{equation}\label{eqn:11.06.06}
  P ^ { \prime } = - M ^ { \prime } v
\end{equation}
利用式\eqref{eqn:11.06.02},得到
\begin{equation}\label{eqn:11.06.07}
  \frac { m _ { 1 } } { m _ { 2 } } = - \frac { u _ { 2 } } { u _ { 1 } }
\end{equation}
% 344.jpg
\clearpage\noindent
另一方面,由式\eqref{eqn:11.06.04}及式\eqref{eqn:11.06.06},可得
\begin{equation}\label{eqn:11.06.08}
  m _ 1 ^ { \prime } u _ 1 ^ { \prime } + m _ 2 ^ { \prime } u _ 2 ^ { \prime } = - \left( m _ 1 ^ { \prime } + m _ 2 ^ { \prime } \right) v
\end{equation}
\begin{align}\label{eqn:11.06.09}
  \beforetext{即}\frac { m _ { 1 } ^ { \prime } } { m _ { 2 } ^ { \prime } } = - \frac { \left( u _ { 2 } ^ { \prime } + v \right) } { \left( u _ { 1 } ^ { \prime } + v \right) }
\end{align}
再由相对论速度合成公式\eqref{eqn:11.05.09},有
\begin{equation*}
  \begin{aligned}
     & u_{1}=\frac{u_{1}^{\prime}+v}{1+{v u_{1}^{\prime}}/{c^{2}}} \\
     & u_{2}=\frac{v_{2}^{\prime}+v}{1+{v u_{2}^{\prime}}/{c^{2}}}
  \end{aligned}
\end{equation*}
将此代入式\eqref{eqn:11.06.09},得
\begin{equation*}
  \frac{m_{1}^{\prime}}{m_{2}^{\prime}}=-\frac{1+{v u_{2}^{\prime}}/{c^{2}}}{1+{v u_{1}^{\prime}}/{c^{2}}} \cdot \frac{u_{2}}{u_{1}}
\end{equation*}
再用式\eqref{eqn:11.06.07}
\begin{equation*}
  \frac{m_{1}^{\prime}}{m_{2}^{\prime}}=\frac{1+{v u_{2}^{\prime}}/{c^{2}}}{1+{v u_{1}^{\prime}}/{c^{2}}} \cdot \frac{m_{1}}{m_{2}}
\end{equation*}
\begin{align}\label{eqn:11.06.10}
  \beforetext{即}\frac{m_{1}^{\prime} / m_{1}}{m_{2}^{\prime} / m_{2}}=\frac{1+v u_{2}^{\prime} / c^{2}}{1+v u_{1}^{\prime} / c^{2}}
\end{align}
注意下列公式
\begin{equation*}
  \begin{aligned}
    \sqrt{1-\frac{u_{1}^{2}}{c^{2}}} & =\sqrt{1-\frac{\left(u_{1}+v\right)^{2}}{c^{2}\left(1+v u_{1} / c^{2}\right)^{2}}}                 \\
                                     & =\frac{\sqrt{1-v^{2} / c^{2}} \cdot \sqrt{1-u_{1}^{\prime 2} / c^{2}}}{1+v u_{1}^{\prime} / c^{2}}
  \end{aligned}
\end{equation*}
% 345.jpg
\clearpage\noindent
\begin{align*}
  \beforetext{即}1+\frac{v u_{1}^{\prime}}{c^{2}}=\frac{\sqrt{1-u_{1}^{\prime 2} / c^{2}}}{\sqrt{1-u_{1}^{2} / c^{2}}} \cdot \sqrt{1-\frac{v^{2}}{c^{2}}}
\end{align*}
类似地
\begin{equation*}
  1+\frac{v u_{2}^{\prime}}{c^{2}}=\frac{\sqrt{1-u_{2}^{\prime 2} / c^{2}}}{\sqrt{1-u_{2}^{2} / c^{2}}} \cdot \sqrt{1-\frac{v^{2}}{c^{2}}}
\end{equation*}
将上两式代入式(11.6.10),得到
\begin{equation}\label{eqn:11.06.11}
  \frac{m_{1}^{\prime} / m_{1}}{m_{2}^{\prime} / m_{2}}=\frac{\sqrt{1-u_{1}^{2} / c^{2}} / \sqrt{1-u_{1}^{\prime 2} / c^{2}}}{\sqrt{1-u_{2}^{2} / c^{2}} / \sqrt{1-u_{2}^{\prime 2} / c^{2}}}
\end{equation}
这就是利用相对性原理导出的,对于任意两个质点的对$ K $及$ K ' $
的质量及速度之间必须保持的关系。在上式中含下标1及2各
量,以及带上撇及不带撇各量的对称位置是很明显的,只要我
们取质点的质量与其速度间有如下关系,就可满足式\eqref{eqn:11.06.11}
\begin{equation*}
  m = \frac { \alpha } { \sqrt { 1 - u ^ { 2 } / c ^ { 2 } } }
\end{equation*}
其中$ \alpha $为常数。为了确定常数$ \alpha $,我们利用当速度$ u $小时,$ m $应
趋于牛顿力学中所用的质量,即静止质量$ m _ 0 $,所以应有$ \alpha = m _ { 0 } $,
故
\begin{equation}\label{eqn:11.06.12}
  m = \frac { m _ { 0 } } { \sqrt { 1 - u ^ { 2 } / c ^ { 2 } } }
\end{equation}
这是相对论力学的重要结论之一。即质点的质量并不是不变的
量,而是与质点的运动状态有关的,质点速度越大,它的质量越
大。严格说只有当速度为零时,质量才等于静止质量,实际上只
要$ u \ll c $,就有$ m \approx m _ 0 $,这就是牛顿运动方程中把质量看作常量的
根据。

\end{document}
